% 2.3
\subsection{無順序木パターン}
\textbf{無順序木(unordered tree)}とは,後述する順序木と異なり同じ親を持つ頂点に順序関係(兄弟関係)を持たない木のことである.図\ref{fig:sibling_relationship}の2つの木は描き方は異なるが同じ木と扱われる.

% 定義1
\begin{define}{\bf 無順序木における変数}\par
  $\ell$を1以上の整数とする.$T=(V_T,E_T)$の変数とは,次の条件(1)--(3)を満たす$V_T$に含まれる頂点のリスト$h=[v_0,v_1,\ldots,v_{\ell}]$である.
  \begin{enumerate}
    \item[(1)] 頂点$v_0$は子を持つ.
    \item[(2)] $v_1,v_2,\ldots,v_{\ell}$ $(v_i \in V_T, 1\leq i\leq \ell)$は$v_0$の子である.% 順序木のときは「の連続した子である」
    \item[(3)] 変数$h$にはランクが$\ell+1$の変数ラベル$x\in X$がラベル付けされている.
  \end{enumerate}
  $v_0$を変数$h$の親ポート,$v_1,v_2,\ldots,v_{\ell}$を変数$h$の子ポートと呼ぶ.変数の次元とは,その変数に含まれる頂点数のことである.従って,変数$h=[v_0,v_1,\ldots,v_{\ell}]$の次元は$\ell+1$である.\par
  本論文では,変数を図\ref{fig:variable}のように四角で表す.四角に囲まれた文字は変数ラベルを表し,その変数が含む頂点を四角と数字付きの線で結ぶ.数字は変数におけるその頂点の順番を表す.線上の数字は,子ポートの数が1のときは省略する.
  図\ref{fig:variable}では変数ラベル$x$を持つ変数$h=[v_0,v_1,v_2,v_3]$を表す.$h$の次元は4である.
\end{define}

% 図2.2
\begin{figure}[tb]
  \centering
  \includegraphics[scale=0.28]{fig-variable.eps}
  \caption{変数ラベル$x$を持つ変数$h=[v_0,v_1,v_2,v_3]$}\label{fig:variable}
\end{figure}

% 定義2
\begin{define}{\bf 無順序項木パターン(unordered term tree pattern)}\par
  $T=(V_T,E_T)$を無順序木とし,$H_T$を$T$の変数の集合とする.ただし,任意の異なる2つの変数$h_1,\,h_2\in H_T$に対して,$h_1$と$h_2$に共通して現れる頂点は高々1つであるとする.無順序項木パターンとは,次の条件(1)--(4)を満たす3つ組$t=(V_t,E_t,H_t)$である:
  \begin{enumerate}
    \item[(1)] $V_t=V_T$,
    \item[(2)] $E_t=E_T\setminus \left(\bigcup_{[v_0,v_1,\ldots,v_{\ell}]\in H_T}\{(v_0,v_i)\in E_T\mid 1\leq i\leq \ell\}\right)$,
    \item[(3)] $H_t=H_T$,
    \item[(4)] $V_t$に属す頂点の頂点ラベル,$E_t$に属す辺の辺ラベルは,それぞれに対応する$V_T$の頂点ラベル,$E_T$の辺ラベルと同じである.
  \end{enumerate}
\end{define}

% 定義2
\begin{define}{\bf 無順序項木パターンの束縛と代入}\par
  $x\in X$をランク$\ell+1$の変数ラベル,$g$を$\ell+1$個以上の頂点を持つ無順序項木パターンとする.$u_0$を$g$の根とし,$u_1,\ldots,u_{\ell}$を$g$の根以外の互いに異なる$\ell$個の頂点とし,$\sigma=[u_0,u_1,\ldots,u_{\ell}]$とする.このとき,形式$x:=[g,\sigma]$を$g$の$\sigma$による$x$への\textbf{束縛(binding)}または単に$x$の束縛という.
  $f=(V_{f},E_{f},H_{f})$を無順序項木パターンとする.ランク$\ell+1$の変数ラベル$x$を持つ変数が$f$に存在するとき,その変数を$h_1,\ldots,h_k$ $(k\geq 0)$とする.束縛$x:=[g,\sigma]$を変数$h_1,\ldots,h_k$に次のように同時に適用して得られる無順序項木パターンを$f\{x:=[g,\sigma]\}$と書く:
  \begin{enumerate}
    \item[(1)] $g_1,\ldots,g_k$を無順序項木パターン$g$と同型な無順序項木パターンとする.
    \item[(2)] 変数$h_j=[v_0^{(j)},v_1^{(j)},\ldots,v_{\ell}^{(j)}]$ $(1\leq j\leq k)$に関して,$H_f$から変数$h_j$を削除し,$g$の頂点$u_0,u_1,\ldots,u_{\ell}$に対応する$g_j$の頂点$u_0^{(j)},u_1^{(j)},\ldots,u_{\ell}^{(j)}$を,この順序で変数$h_j$の頂点$v_0^{(j)},v_1^{(j)},\ldots,v_{\ell}^{(j)}$と同一視する.同一視後の頂点ラベルは,頂点$v_0^{(j)},v_1^{(j)},\ldots,v_{\ell}^{(j)}$の頂点ラベルを継承する.
  \end{enumerate}
  \textbf{代入(substitution)}とは,$x_i\,(1\leq i\leq n)$が$X$の異なる変数ラベルであるような束縛の有限集合$\theta=\{x_1:=[g_1,\sigma_1],\ldots,x_n:=[g_n,\sigma_n]\}$である.
  ただし,$g_{1},\ldots,g_{n}$には変数ラベル$x_1,\ldots,x_n$を持つ変数が現れないと仮定する.
  代入$\theta$による$f$の\textbf{インスタンス(instance)}とは,$f$に対して,$\theta$の全ての束縛$x_i:=[g_i,\sigma_i]$ $(1\leq i\leq n)$を適用して得られる無順序項木パターンである.代入$\theta$による$f$のインスタンスを$f\theta$と書く.

  無順序木(順序木) $T_1$と$T_2$が無順序木(順序木)として同型であるとき,$T_1\equiv T_2$と書く.

  % 図2.3
  \begin{figure}[tb]
    \centering
    \includegraphics[scale=0.28]{fig-lutp.eps}
    \caption{線形無順序木パターン$t$と無順序木$t\theta \equiv T$を示す.無順序木の場合,代入後の$T$と下に示す$T_1,T_2,\ldots ,T_n$は同型である.}\label{fig:lutp} % 無順序木パターン$t$と無順序木$t\theta\equiv T$
  \end{figure}

  $\Sigma,\Lambda$をそれぞれ頂点ラベルの集合,辺ラベルの集合とする無順序木の全体を${\cal UT}_{\Sigma,\Lambda}$とする.また,$\Sigma,\Lambda,X$をそれぞれ頂点ラベルの集合,辺ラベルの集合,変数ラベルの集合とする無順序項木パターンの全体を${\cal UTTP}_{\Sigma,\Lambda,X}$とする.次のように${\cal UTTP}_{\Sigma,\Lambda,X}$の部分集合を定める:
  \begin{align*}
    {\cal LUTTP}_{\Sigma,\Lambda,X} & = \{t=(V_t,E_t,H_t)\in{\cal UTTP}_{\Sigma,\Lambda,X}\mid\\
    & \hspace*{12pt}\forall h_1,\,h_2\in H_t\,(h_1\not= h_2)\mbox{に対して,}\\
    & \hspace*{12pt}\mbox{変数}h_1\mbox{と}h_2\mbox{の変数ラベルは異なる}\},\\
    {\cal LUTP}_{\Sigma,\Lambda,X} & = \{t=(V_t,E_t,H_t)\in{\cal LUTTP}_{\Sigma,\Lambda,X}\mid\\
    & \hspace*{12pt}\forall h\in H_t\mbox{の次元は2であり,かつ}\\
    & \hspace*{12pt}\mbox{$h$の子ポートは葉である}\}.
  \end{align*}
  
  \noindent
  ${\cal LUTTP}_{\Sigma,\Lambda,X}$に属す無順序項木パターンを\textbf{線形無順序項木パターン(linear unordered term tree pattern)}と呼ぶ.また,${\cal LUTP}_{\Sigma,\Lambda,X}$に属す線形無順序項木パターンを\textbf{線形無順序木パターン(linear unordered tree pattern)}と呼ぶ.

  以降,無順序木$T=(V_T,E_T)$を,変数を持たない無順序項木パターン$T=(V_T,E_T,\emptyset)$とみなす.
\end{define}

% 定義4
\begin{define}{\bf 無順序項木パターン言語}\par
  無順序項木パターン$t\in {\cal UTTP}_{\Sigma,\Lambda,X}$に対して,$t$の無順序項木パターン言語$L(t)\subseteq {\cal UT}_{\Sigma,\Lambda}$を次のように定義する:
  $$L(t)=\{t\theta\in {\cal UT}_{\Sigma,\Lambda}\mid \mbox{$\theta$は$t$の変数への任意の代入}\}.$$
  \noindent
  線形無順序木パターン$t$に対して,$T\in L(t)$となる無順序木$T$の例を図\ref{fig:lutp}にあげる.
\end{define}


無順序項木パターン$t$と無順序木$T$に対して,代入$\theta$が存在して$t\theta\equiv T$となるとき,$t$は$T$にマッチするという.
%無順序項木パターン照合問題は次のように定義される決定問題である:
%
%\medskip
%\noindent
%\textbf{無順序項木パターン照合問題}(${\cal MP}$-${\cal UTTP}$)\\
%\textbf{入力}: 無順序項木パターン$t$と無順序木$T$;\\
%\textbf{問題}: $t$は$T$にマッチするか?
%\medskip
%
%上記問題と同様に,無順序項木パターンを線形無順序項木パターンに限定した\textbf{線形無順序項木パターン照合問題}(${\cal MP}$-${\cal LUTTP}$)を定める.無順序項木パターン照合問題は,$t$が線形無順序項木パターンであったとしても,次元4以上の変数が存在すればNP完全である\cite{shoudai-ieice2018}.一方,全ての変数の次元が2であれば,線形無順序項木パターン$t$に対する無順序項木パターン照合問題には入力サイズの多項式時間アルゴリズムが存在する\cite{shoudai-ieice2018}.従って,線形無順序木パターン$t\in{\cal LUTP}_{\Sigma,\Lambda,X}$に対する無順序項木パターン照合問題は入力サイズの多項式時間で計算可能である.
%無順序項木パターンに対する無矛盾性問題を次のように定義する.\par
%
%\medskip
%\noindent
%\textbf{無順序項木パターンに対する無矛盾性問題}(${\cal CP}$-${\cal UTTP}$)\\
%\textbf{入力}: 無順序木の有限集合$S_+\subseteq {\cal UT}_{\Sigma,\Lambda}$と$S_-\subseteq {\cal UT}_{\Sigma,\Lambda}$,ただし$S_+\cap S_-=\emptyset$.\\
%\textbf{問題}: $S_+\subseteq L(t)$かつ$S_-\cap L(t)=\emptyset$を満たす無順序項木パターン$t\in {\cal UTTP}_{\Sigma,\Lambda,X}$は存在するか?
%\medskip
%
%上記問題と同様に,無順序項木パターンを線形無順序項木パターンに限定した\textbf{線形無順序項木パターンに対する無矛盾性問題}(${\cal CP}$-${\cal LUTTP}$)を定める.Miyanoら\cite{miyano-ngc2000}は正規パターンに対して定義された無矛盾性問題がNP完全であることを示した.その結果から,${\cal CP}$-${\cal LUTTP}$がNP完全であることが導かれる.