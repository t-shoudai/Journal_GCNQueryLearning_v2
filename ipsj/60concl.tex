% 6
%\chapter{結論}
\section{まとめと今後の課題}
本論文では,線形無順序木パターンに対する質問学習モデルのオラクルとしてグラフ畳み込みネットワーク(Graph Convolutional Network, GCN)を採用した新たな質問学習モデルを提案した.このモデルは,従来の完全な教師(オラクル)を仮定した手法とは異なり,高精度な深層学習モデルであるGCNを活用し,データの構造的特徴を効率的に発見することを目的としている.さらに,本研究では,質問学習モデルで使用されるアルゴリズムについて検討を行い,既存手法よりも高速な処理を可能とする新たなアルゴリズムを提案した.

第1章では,まず背景として,近年,グラフ構造を持つデータが飛躍的に増加しており,これらのデータを活用した研究が盛んに行われている現状について述べた.これらのデータ活用手法の一つとして,本研究ではGCNに注目した.GCNは,グラフ構造データを対象とした機械学習手法であり,高精度な分類が可能であることが報告されている.また,計算論的学習理論における機械学習モデルの一つである質問学習にも着目した.質問学習モデルの中でも,本研究では,学習対象が目標概念に属すか否かを$Yes$または$No$で応答する所属性質問を利用した.この両手法を順序項木パターンに応用した先行研究として,小田ら\cite{oda-ai2022}による研究があることを述べた.この研究では,順序木データを対象にGCNを用いて学習を行い,さらにそのGCNの予測根拠を質問学習モデルを通じて順序項木パターンとして明らかにする手法を提案し,実験的にその有効性を示している.本研究では,対象を無順序木に拡張し,(1)二値分類問題,(2)無矛盾性問題,(3)可視化問題の3つの計算問題を解析することで,GCNと質問学習モデルを統合した協調モデルの有効性を示すことを目的とした.さらに,質問学習モデルにおけるアルゴリズムについても検討を行い,既存手法に比べて高速な処理を可能とする新たなアルゴリズムを提案した.

第2章では,本研究の準備として,本論文で使用する基本的な用語と概念を整理した.離散数学におけるグラフは,頂点集合を$V$,辺集合を$E$とするとき,$G=(V,E)$で表される集合構造である.本研究では,グラフの一種である木を対象とし,特別な頂点である根(root)を1つ持ち,閉路を含まない構造としての根付き木に注目した.木は,性質に基づいて大きく2つの種類に分類される.一つは,兄弟関係を持たず,子の順序が定義されていない無順序木であり,もう一つは,兄弟関係を持ち,子の順序が明確に定義される順序木である.

第3章では,線形無順序木パターンを質問学習に適応する方法について述べた.まず,線形無順序木パターンに関連する計算問題の計算量について議論し,照合問題(${\cal LUTP}$-${\cal MP}$)と無矛盾性問題(${\cal LUTP}$-${\cal CP}$)がNP完全であることを示した.これらの証明は,充足可能性問題(3-SAT)からそれぞれの問題への多項式時間帰着を構築することで行った.次に,Amoth\cite{amoth-ml2001}によって提案されたアルゴリズムを基に,線形無順序木パターンの同定を可能とする新たなアルゴリズム${\cal LUTP}$-${\cal QUERY}_{{\cal O}(t)}^{Square}$を提案した.このアルゴリズムは,入力として与えられる無順序木の頂点数を$N$とした場合,$O(N^2)$回の所属性質問を用いて元の線形無順序木パターンを同定可能である.さらに,計算効率を向上させるため,$O(N)$回の所属性質問で同定を実現するアルゴリズム${\cal LUTP}$-${\cal QUERY}_{{\cal O}(t)}^{Linear}$を提案した.これらのアルゴリズムについて理論的解析および計算機実験を通じて詳細に検証し,その正当性と有効性を証明した.

第4章では,本論文のタイトルにも示されている,線形無順序木パターンに対するグラフ畳み込みネットワーク(GCN)をオラクルとする質問学習モデルについて述べた.まず,GCNの学習に使用するPythonの機械学習ライブラリであるPyTorch Geometric(PyG)に実装されている畳み込み層であるGCNConv,GraphConv,RGCNConvについて,それぞれの畳み込み方法の違いを数式を用いて説明した.これらのグラフ畳み込み層を利用して,GCNをオラクルとする線形無順序木パターンに対する所属性質問を提案した.この手法では,質問学習モデルにおける完全な教師(オラクル)をGCNに置き換えることで,協調モデルを構築する.この協調モデルは,不完全な教師である場合でも,元の線形無順序木パターンを発見することを目的としている.次に,提案する協調モデルの有効性を確認するための計算機実験について述べた.評価は,データセット$S$を対象として以下の3つの計算問題に基づき,F値を用いて行った.(1)二値分類問題:無順序木の特徴量を用いて,データセット$S$に属す無順序木が元のクラスに分類可能かを検証する.(2)無矛盾性問題:データセット$S$をクラス分類するための線形無順序木パターンが存在するか否かを決定する.(3)可視化問題:一般にブラックボックスである深層学習モデルの予測根拠を,線形無順序木パターンを用いて明らかにする.実験結果から,非常に高い精度でデータセットを分類するGCNを学習できることが確認された.しかし,新しいデータセットに対しては,学習データとの差が大きく,過学習が発生している可能性が示唆された.一方で,提案する協調モデルは高い精度で分類する線形無順序木パターンを一定程度発見できた.これは,学習データと新しいデータセット間の差が小さく,協調モデルが過学習を起こしにくいことを示している.さらに,グラフ畳み込み層と質問学習アルゴリズムの性能の違いを検討した.結果として,GraphConvを使用した場合はアルゴリズム${\cal LUTP}$-${\cal QUERY}_{{\cal O}(t)}^{Square}$の精度が高く,RGCNConvを使用した場合はアルゴリズム${\cal LUTP}$-${\cal QUERY}_{{\cal O}(t)}^{Linear}$の精度が高いことが確認された.これは,GraphConvが隣接する頂点数に基づいて特徴を集約するのに対し,RGCNConvが辺ラベルの種類に基づいて特徴を集約するという両者の特性の違いによる結果であると考察した.さらに,計算機実験ではランダムに生成した線形無順序木パターンを用いており,協調モデルによって出力される線形無順序木パターンの精度が高いものと低いものが存在することが確認された.そこで,精度と各要素間の相関関係について調査を行ったが,いずれの結果にも有意な相関は見られなかった.

今後の課題として,協調モデルに対して無順序木の項木パターンへの拡張が挙げられる.東山ら\cite{higashiyama-hinokuni2024}は,順序木に内部変数を含む線形順序項木パターンを用いた協調モデルを提案し,非常に高精度なデータ分類を実現する線形順序項木パターンの獲得に成功している.しかし,この方法を線形無順序項木パターンに適用するためには,新たな課題を解決する必要がある.無順序木は兄弟関係に順序がない特性を持つため,従来のアルゴリズムや本研究で提案した${\cal LUTP}$-${\cal QUERY}_{{\cal O}(t*)}^{Linear}$などのアルゴリズムをそのまま使用することは困難である.この特性により,内部変数を含む線形無順序項木パターンの構造を効率的に扱うアルゴリズムの設計が極めて難しい.したがって,無順序木の特性を踏まえた新たな質問学習アルゴリズムの開発が不可欠である.またそれに加え,実世界で得られるグラフデータに対して,本研究で提案したグラフ畳み込みネットワーク(GCN)と質問学習モデルの協調モデルを適用し,具体的なパターン発見手法を確立することがあげられる.この際,グラフデータの複雑な構造を表現する手段として,形式グラフ文法や形式グラフ体系をグラフパターンの記述手法として採用することが有効であると考えられる.さらに,それらを活用したGCNと質問学習モデルの協調モデルを設計・解析し,実世界の多様なグラフデータにおけるパターン発見の精度向上や新たな知見の抽出に寄与することがあげられる.
