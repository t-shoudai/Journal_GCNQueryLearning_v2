% 4.1
\subsection{グラフ畳み込みネットワーク(GCN)}

%\subsection{使用する中間層}
%グラフ構造におけるノード間の情報伝播を実現するための手法であり,隣接行列とその次数行列を用いてノードの特徴量を集約する.GraphConvの主な特徴は,隣接行列を二重に正規化する点にある.このアプローチは,隣接ノードの特徴量を単に加重平均するのではなく,次数の平方根で正規化することによって,グラフ構造をより精緻に反映させることを目的としている.まず,隣接行列を次数行列の逆平方根で左および右に正規化し,その後,隣接ノードの特徴量を加重平均することで,各ノードの新たな特徴量が算出される.この手法は,グラフ上のノードが持つ情報を隣接ノードの情報と効果的に融合させることができるため,グラフの構造を反映した学習が可能となる.GraphConvの計算式は以下のように表される:
\textbf{GCNConv}\footnote{\url{https://pytorch-geometric.readthedocs.io/en/2.5.2/generated/torch_geometric.nn.conv.GCNConv.html}}\cite{pyg-gcnconv}は,頂点の次数に基づく正規化を行うことで,異なる次数を持つ頂点間での特徴ベクトルを正規化する役割を果たす.この正規化により,次数の差異が学習に与える影響を軽減し,頂点間の関係性をより適切に反映した特徴抽出が可能となる.これにより,グラフ全体にわたる情報の伝播が効率的に行われるよう設計されている.式(\ref{eq:gcnconv})に頂点$v$の特徴ベクトル$x_v$を更新するグラフ畳み込み層GCNConvの計算式を示す.ここで,$\textbf{A}$を頂点数$n$~$(n\geq 1)$の入力グラフの隣接行列とする.$\hat{\textbf{A}}=\textbf{A}+\textbf{I}$は,自己ループを持つ隣接行列を表し,$\hat{\textbf{D}}$は,$\hat{D}_{ij}=\sum_{k=1}^{n}\hat{A}_{ik}$~$(i=j)$,$\hat{D}_{ij}=0$~$(i\not=j)$である対角次数行列を表す.ただし,$\hat{A}_{ij}$と$\hat{D}_{ij}$はそれぞれ$\hat{\textbf{A}}$と$\hat{\textbf{D}}$の$(i,j)$成分を表す.すなわち,$\hat{\textbf{D}}^{-\frac{1}{2}}\hat{\textbf{A}}\hat{\textbf{D}}^{-\frac{1}{2}}$で正規化が行われる.
\begin{equation}
  \label{eq:gcnconv}
  \bm{x}'_{v} = \hat{\textbf{D}}^{-\frac{1}{2}}\hat{\textbf{A}}\hat{\textbf{D}}^{-\frac{1}{2}}\bm{x}_{v}\Theta.
\end{equation}

\textbf{GraphConv}\footnote{\url{https://pytorch-geometric.readthedocs.io/en/2.5.3/generated/torch_geometric.nn.conv.GraphConv.html}}\cite{pyg-graphconv}は,特徴量を隣接頂点と自己ループから集約して更新する.つまり,全ての隣接頂点の特徴ベクトルは単純な和として計算される.そのため,頂点の次数が高いほど,より多くの情報を集約することになる.この手法により,次数の影響がそのまま情報集約の範囲に反映され,頂点の重要度に基づいた学習が行われる.式(\ref{eq:graphconv})に頂点$v$の特徴ベクトル$x_v$を更新するグラフ畳み込み層GraphConvの計算式を示す.ここで,$x'_v$は更新した特徴ベクトルを,$\textbf{W}_1,\textbf{W}_2$は重み行列を,$N(v)$は頂点$v$における隣接頂点の集合を表す.また,$e_{\{u,v\}}$は辺$\{u,v\}$に与えられた数値を表す.
\begin{equation}
  \label{eq:graphconv}
  \bm{x}'_{v}=\textbf{W}_{1}\bm{x}_{v}+\textbf{W}_{2}\sum_{v'\in N(v)}e_{\{u,v\}}\bm{x}_{u}.
\end{equation}

\textbf{RGCNConv}\footnote{\url{https://pytorch-geometric.readthedocs.io/en/2.6.1/generated/torch_geometric.nn.conv.RGCNConv.html}}\cite{pyg-rgcnconv}は,関係性を特徴としてGCNConvを拡張した手法である.$R$を辺ラベルの集合とする.ここでは辺ラベルが関係性を表すとする.各辺ラベル$r\in R$に対して専用の重み行列を持つことが特徴である.辺ラベルごとに異なる重み行列を使用し,特徴量を集約して更新することにより,新たな特徴ベクトルを生成する.この過程により,異なる関係性に基づく意味的な差異を捉えることができ,関係性ごとに情報を適切に集約することが可能となる.式(\ref{eq:rgcnconv})に頂点$v$の特徴ベクトル$x_v$を更新するグラフ畳み込み層RGCNConvの計算式を示す.ここで,$x'_v$は更新した特徴ベクトルを,$\textbf{W}_1,\textbf{W}_r$~$(r\in R)$は重み行列を,$N_r(v)$は頂点$v$における辺ラベル$r\in R$ごとに異なる隣接頂点の集合を表す.
\begin{equation}
  \label{eq:rgcnconv}
  \bm{x}'_{v}=\textbf{W}_{1}\bm{x}_{v}+\sum_{r\in R}\left(\sum_{v'\in N_{r}(v)}\frac{1}{|N_{r}(v)|}\textbf{W}_{r}\bm{x}_{v'}\right).
\end{equation}