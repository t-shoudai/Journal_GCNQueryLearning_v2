% 3.3
\subsection{$O(N^2)$回の所属性質問による質問学習アルゴリズム}
線形無順序木パターン$t_*\in {\cal LUTP}_{\Sigma,\Lambda,X}$の所属性質問に答えるオラクルを${\cal O}(t_*)$と書く.アルゴリズム${\cal LUTP}$-${\cal QUERY}_{{\cal O}(t_{\ast})}^{Square}$ (アルゴリズム\ref{alg:lutp-query-square})は,Amoth\cite{amoth-ml2001}により提案されたアルゴリズムを応用したものである.ただし,Amothのアルゴリズムは,頂点を変数とみなす無順序木構造パターンに対するアルゴリズムであるので,辺を変数とみなす本論文の線形無順序木パターンでは,もし$|\Sigma|\not=\infty$かつ$|\Lambda|\not=\infty$ならば,そのままでは正しく動作しない.

$t_*$にマッチする無順序木$T$を入力とする.このとき,${\cal LUTP}$-${\cal QUERY}_{{\cal O}(t_{\ast})}^{Square}$は,もし$|\Sigma|=\infty$または$|\Lambda|=\infty$ならば,線形無順序木パターン$t\in {\cal LUTP}_{\Sigma,\Lambda,X}$で$L(t_*)=L(t)$となる$t$を,${\cal O}(t_*)$に対して$O(N^2)$回の所属性質問を行なって正しく出力することができる.ただし,$N$は$T$の頂点数である.そのときアルゴリズム{\sc Make\_Min\_Pos\_Tree\_Sqr} (アルゴリズム\ref{alg:lutp-query-square-step1})は$t_*$にマッチする$T$を入力とした場合,その出力として,$t_*$と同じ頂点数を持ち,かつ頂点数が最小となる正例を返す.まず$T$のうち,根から最も深い頂点とその頂点に接続する辺を削除し,これを$T'$とする.次に,オラクル${\cal O}(t_*)$に$T'$を問い合わせ,${\cal O}(t_*)(T')=Yes$であれば,$T$を$T'$で更新し,${\cal O}(t_*)(T')=No$であれば,削除した頂点と辺を元に戻す.さらに,処理がいくつか進んだ後に頂点や辺の削除が成功した場合,最も深い頂点から再び順に処理を行う.この操作を再帰的に繰り返し,$T$が変化しなくなると処理を終了する.この処理により,入力された$T$が$t_*$の構造を崩すことなく正確に反映しながら特定していく.その後,頂点数最小となった無順序木$T'$を入力とするとき,もし$|\Sigma|=\infty$または$|\Lambda|=\infty$ならば,アルゴリズム\ref{alg:lutp-query-square-step2} {\sc Variable\_Specify\_Sqr}は$t_*$と同型な$t$を出力する.

具体的には,正例の子の数を$c$とするとき,$T$に現れない頂点ラベルまたは辺ラベルを貼り付けた$c+1$個の辺と頂点(熊手型)を葉に接続し,${\cal O}(t_*)(T')=Yes$であればその葉を端点とする辺を変数に置き換え,${\cal O}(t_*)(T')=No$であれば変数に置き換えず,辺のままとする.このように熊手型を代入することにより変数か否かを決定する.しかし,もし$|\Sigma|\not=\infty$かつ$|\Lambda|\not=\infty$ならば,後述する図\ref{fig:counter_leaf_to_rake}のように変数を同定できないことがある.

% アルゴリズム1
\begin{algorithm}[tb]
\caption{{\sc Make\_Min\_Pos\_Tree\_Sqr};} \label{alg:lutp-query-square-step1}
\begin{algorithmic}[1]
  \Require{$t_{\ast}\in {\cal LUTP}$の正例$T\in L(t_{\ast})$, $t_{\ast}\in {\cal LUTP}$に対する所属性質問に答えるオラクル${\cal O}(t_{\ast})$;}
  \Ensure{$T\in L(t_{\ast})$;}
  \Function{Make\_Min\_Pos\_Tree\_Sqr}{$T$};
    \State $L\coloneqq T$の葉の集合;
    \While{$L\not= \emptyset$}
      \State $T'\coloneqq T$;
      \State $\ell$を$L$の葉の1つとする;
      \State $V'$から$\ell$,$E'$から$(p(\ell),\ell)$を削除,ただし$V'$と$E'$は$T'$の頂点集合と辺集合である;
      \If{${\cal O}(t_{\ast})(T')="Yes"$}
        \State $T\coloneqq T'$;
        \State $L\coloneqq T$の葉の集合;
      \Else
        \State $L$から$\ell$を削除;
      \EndIf
    \EndWhile
    \State \Return $T$;
  \EndFunction
\end{algorithmic}
\end{algorithm}

% アルゴリズム2
\begin{algorithm}[tb]
\caption{{\sc Variable\_Specify\_Sqr};} \label{alg:lutp-query-square-step2}
\begin{algorithmic}[1]
  \Require{最小正例$T\in L(t_{\ast})$, $t_{\ast}\in {\cal LUTP}$に対する所属性質問に答えるオラクル${\cal O}(t_{\ast})$;}
  \Ensure{$t\in {\cal LUTP}$;}
  \Function{Variable\_Specify\_Sqr}{$T$}
    \State $L\coloneqq T$の葉の集合; \,$c\coloneqq \max(\deg(T))$;
    \State$t\coloneqq T$;
    \For{$\ell\in L$}
      \State $T'\coloneqq T$の$\ell$に$c+1$個の子を接続;
      \If{${\cal O}(t_{\ast})(T')="Yes"$}
        \State $T\coloneqq T'$;
        \State $T'$の$(p(\ell),\ell)$に対応する$t$の辺ラベルを変数ラベルに更新;
      \EndIf
    \EndFor
    \State \Return $t$;
  \EndFunction
\end{algorithmic}
\end{algorithm}

% アルゴリズム3
\begin{algorithm}[tb]
\caption{${\cal LUTP}$-${\cal QUERY}_{{\cal O}(t_{\ast})}^{Square}$} \label{alg:lutp-query-square}
\begin{algorithmic}[1]
  \Require{$t_{\ast}\in {\cal LUTP}$の正例$T\in L(t_{\ast})$, $t_{\ast}\in {\cal LUTP}$に対する所属性質問に答えるオラクル${\cal O}(t_{\ast})$;}
  \Ensure{$t\in {\cal LUTP}$;}
  \State $T\coloneqq $ \Call{Make\_Min\_Tree\_Sqr}{$T$};
  \State $t\coloneqq $ \Call{Variable\_Specify\_Sqr}{$T$};
\end{algorithmic}
\end{algorithm}